\documentclass[lang=cn,newtx,10pt,scheme=chinese]{elegantbook}
\usepackage{realboxes}
\usepackage{forest}
\usepackage{import}
\usepackage{adjustbox}

\title{使用C语言构建终端文本编辑器}
\author{左元}

\setcounter{tocdepth}{3}

\cover{cover.pdf}

% 本文档命令
\usepackage{array}
\newcommand{\ccr}[1]{\makecell{{\color{#1}\rule{1cm}{1cm}}}}

% 修改标题页的橙色带
\definecolor{customcolor}{RGB}{32,178,170}
\colorlet{coverlinecolor}{customcolor}
\usepackage{cprotect}

\newtcolorbox{marker}[1][]{enhanced,
  before skip=2mm,after skip=3mm,
  boxrule=0.4pt,left=5mm,right=2mm,top=1mm,bottom=1mm,
  colback=yellow!50,
  colframe=yellow!20!black,
  sharp corners,rounded corners=southeast,arc is angular,arc=3mm,
  underlay={%
    \path[fill=tcbcolback!80!black] ([yshift=3mm]interior.south east)--++(-0.4,-0.1)--++(0.1,-0.2);
    \path[draw=tcbcolframe,shorten <=-0.05mm,shorten >=-0.05mm] ([yshift=3mm]interior.south east)--++(-0.4,-0.1)--++(0.1,-0.2);
    \path[fill=yellow!50!black,draw=none] (interior.south west) rectangle node[white]{\Huge\bfseries !} ([xshift=4mm]interior.north west);
    },
  drop fuzzy shadow,#1}

\tcbuselibrary{listings, skins, breakable}
\usepackage[T1]{fontenc}
\usepackage[ttdefault=true]{AnonymousPro}
\definecolor{pblue}{rgb}{0.13,0.13,1}
\definecolor{pgreen}{rgb}{0,0.5,0}

\newtcblisting[auto counter, number within=chapter]{mycode}[1]{
    breakable,
    enhanced,
    attach boxed title to top right={yshift=-\tcboxedtitleheight},
    boxed title style={
        size=small,colback=gray!50,
        colframe=gray!50,
        sharp corners=downhill,
        arc=.5cm,
        top=1mm,bottom=1mm,left=1mm,right=1mm
    },
    fonttitle=\color{black}\itshape\ttfamily,
    colframe=gray!20,
    top=\tcboxedtitleheight,
    bottom=\tcboxedtitleheight,
    sharp corners=downhill,
    arc=.5cm,
    title={#1},
    listing only,
    listing options={
        escapeinside={(*@}{@*)},
        language=c,
        basicstyle=\fontfamily{AnonymousPro}\selectfont,
        keywordstyle=\bfseries\color{pblue},
        stringstyle=\bfseries\itshape\color{green!40!black},
        commentstyle=\bfseries\itshape\color{black!60},
        % Line numbers
        xleftmargin={0.75cm},
        numbers=left,
        stepnumber=1,
        firstnumber=1,
        numberfirstline=true,
        showspaces=false,
        showtabs=false,
        breaklines=true,
        showstringspaces=false,
        tabsize=1,
        emph={
            gpio_config_t, for, uint8_t, TextView, Toast, Button, EditText, ImageView, Typeface, Intent, WebView, WebSettings, SwipeRefreshLayout, RelativeLayout, Animation, AlertDialog, SharedPreferences, Editor, ToggleButton, CardView, LinearLayout, gradient, shape,
        },
        emphstyle={\bfseries\color{pblue}},
        frame=l
    }
}

\begin{document}

\maketitle
\frontmatter

\tableofcontents

\mainmatter

\chapter{设置}

\begin{mycode}{第1步: kilo.c}
int main() {
  return 0;
}
\end{mycode}

编写 Makefile 文件。

\begin{mycode}{第2步: Makefile}
kilo: kilo.c
    $(CC) kilo.c -o kilo -Wall -Wextra -pedantic -std=c99
\end{mycode}

Makefile 中一定要使用制表符. 命令中的参数:

\begin{itemize}
    \item \Colorbox{lightgrey}{\lstinline{$(CC)}} 是一个 make 会展开的变量, 默认是 cc.
    \item \Colorbox{lightgrey}{\lstinline{-Wall}} 代表``所有警告'', 并让编译器在看到程序中的代码时向你发出警告, 这些代码在技术上可能没有错误, 但被认为是 C 语言的错误或有问题的用法, 例如在初始化变量之前使用变量.
    \item \Colorbox{lightgrey}{\lstinline{-Wextra}} 和 \Colorbox{lightgrey}{\lstinline{-pedantic}} 会打开更多警告. 对于本教程中的每个步骤, 如果你的程序能够编译通过, 除了在某些情况下出现``未使用的变量''警告外, 它不应产生任何警告. 如果你收到任何其他警告, 请检查以确保你的代码与该步骤中的代码完全匹配.
    \item \Colorbox{lightgrey}{\lstinline{-std=c99}} 指定我们正在使用的 C 语言标准的确切版本, 即 C99. C99 允许我们在函数内的任何地方声明变量, 而 ANSI C 要求所有变量都在函数或块的顶部声明.
\end{itemize}

使用 make 命令来编译程序. 运行 \Colorbox{lightgrey}{\lstinline{./kilo}} . 然后使用命令 \Colorbox{lightgrey}{\lstinline{echo $?}} 查看程序的返回值.

\chapter{进入原始模式}

\begin{marker}
原始模式: raw mode
\end{marker}

接下来读取用户的按键操作.

\begin{mycode}{第3步: kilo.c}
#include <unistd.h>

int main() {
  char c;
  while (read(STDIN_FILENO, &c, 1) == 1);

  return 0;
}
\end{mycode}

\Colorbox{lightgrey}{\lstinline{read()}} 和 \Colorbox{lightgrey}{\lstinline{STDIN_FILENO}} 来自 <unistd.h> . \Colorbox{lightgrey}{\lstinline{read()}} 从标准输入中读取 1 个字节到变量 c 中, 在 while 循环中一直读取, 直到没有可以读取的字节. read() 返回读取的字节数, 并在到达文件末尾时返回 0.

当运行 \Colorbox{lightgrey}{\lstinline{./kilo}} 时, 终端会连接到标准输入, 因此键盘的输入会被读取到变量 c 中. 但是, 默认情况下终端以 \textbf{规范模式}\footnote{canonical mode} 启动. 在规范模式下, 键盘输入仅在用户按下 \textbf{回车键} 时发送到我们的程序. 这对许多程序都很有用: 它允许用户输入一行文本, 这样可以使用 \textbf{退格键} 来修复错误, 直到文本完全按照想要的方式输入, 最后按 \textbf{回车键} 将其发送到程序. 但它不适用于具有更复杂用户界面的程序, 如文本编辑器. 我们希望在每个按键输入时都对其进行处理, 以便我们可以立即做出响应.

我们想要的是 \textbf{原始模式} . 不幸的是, 没有简单的开关可以将终端设置为原始模式. 原始模式是通过关闭终端中的许多标志位来实现的, 我们将在本章的过程中逐渐做到这一点.

要退出上述程序, 请按 Ctrl-D 以告知 read() 它已到达文件末尾. 或者我们始终可以按 Ctrl-C 以发出立即终止进程的信号.

\section{按下q键退出}

为了演示规范模式是如何工作的, 我们将让程序在读取到用户的 q 按键操作时退出.

\begin{mycode}{第4步: kilo.c}
#include <unistd.h>

int main() {
  char c;
  while (read(STDIN_FILENO, &c, 1) == 1 && c != 'q');

  return 0;
}
\end{mycode}

要退出程序, 必须键入一行包含 q 的文本, 然后按回车键. 程序将一次一个字符地快速读取文本行, 直到读取到 q , 此时 while 循环将停止且程序将退出. q 之后的任何字符都将在输入队列中保持未读状态, 你可能会在程序退出后看到该输入被送入你的 shell .

\section{关闭回显(echo)}

我们可以通过以下方式设置终端的属性.

\begin{enumerate}
    \item 使用 tcgetattr() 将当前属性读入结构体 raw.
    \item 手动修改结构体 raw .
    \item 将修改后的结构体 raw 传递给 tcsetattr() 以写回新的终端属性. 让我们尝试以这种方式关闭 ECHO 功能.
\end{enumerate}

\begin{mycode}{第5步: kilo.c}
#include <termios.h>
#include <unistd.h>

void enableRawMode() {
  struct termios raw;

  tcgetattr(STDIN_FILENO, &raw);

  raw.c_lflag &= ~(ECHO);

  tcsetattr(STDIN_FILENO, TCSAFLUSH, &raw);
}

int main() {
  enableRawMode();

  char c;
  while (read(STDIN_FILENO, &c, 1) == 1 && c != 'q');
  return 0;
}
\end{mycode}

struct termios, tcgetattr(), tcsetattr(), ECHO, TCSAFLUSH 都来自 <termios.h> 头文件.

ECHO 的功能是使你键入的每个键都打印到终端, 因此你可以看到你正在键入的内容. 这在规范模式下很有用, 但当我们试图在原始模式下仔细呈现用户界面时, 它确实会妨碍我们. 所以我们关掉它. 该程序与上一步中的程序执行相同的操作, 只是不打印你输入的内容. 如果你曾经不得不在终端输入密码, 例如在使用 sudo 时, 你可能会熟悉这种模式.

程序退出后, 在 shell 里面, 你可能会发现终端仍然没有回显你键入的内容. 别担心, 它仍然会监听你输入的内容. 只需按 Ctrl-C 开始新的一行输入到你的 shell, 然后输入 reset 并按回车键. 在大多数情况下, 这会将你的终端重置为正常. 如果失败, 可以随时重新启动终端模拟器. 我们将在下一步中解决整个问题.

termios 可以通过 tcgetattr() 将终端属性读入结构体 raw . 修改终端的属性之后, 可以使用 tcsetattr() 将修改后的属性应用到终端. TCSAFLUSH 参数指定何时应用更改: 在这种情况下, 它等待所有待处理的输出被写入终端, 并丢弃任何尚未读取的输入.

c\_lflag 字段用于``本地标志位''. 其他标志位字段是 c\_iflag(输入标志位), c\_oflag(输出标志位)和 c\_cflag(控制标志位), 所有这些我们都必须修改以启用原始模式.

ECHO 是一个 bitflag, 二进制表示是: 00000000000000000000000000001000. 我们对它使用按位非运算符(~)来获取 11111111111111111111111111110111. 然后我们将该值与标志位字段按位与, 这会强制标志位字段中的第四位变为 0, 并导致每隔一位保留当前值. 像这样翻转位在 C 语言中很常见.

\section{退出时禁用原始模式}

让我们善待用户, 并在我们的程序退出时恢复他们终端的原始属性. 我们使用变量 orig\_termios 来保存终端的原始属性, 并在程序退出时用于将其应用于终端.

\begin{mycode}{第6步: kilo.c}
#include <stdlib.h>
#include <termios.h>
#include <unistd.h>

struct termios orig_termios;

void disableRawMode() {
  tcsetattr(STDIN_FILENO, TCSAFLUSH, &orig_termios);
}

void enableRawMode() {
  tcgetattr(STDIN_FILENO, &orig_termios);
  atexit(disableRawMode);

  struct termios raw = orig_termios;
  raw.c_lflag &= ~(ECHO);

  tcsetattr(STDIN_FILENO, TCSAFLUSH, &raw);
}

int main() { ... }
\end{mycode}

\section{关闭规范模式}

ICANON 标志位允许我们关闭规范模式. 这意味着我们最终将逐字节读取输入, 而不是逐行读取.

\begin{mycode}{第7步: kilo.c}
#include <stdlib.h>
#include <termios.h>
#include <unistd.h>

struct termios orig_termios;

void disableRawMode() { ... }

void enableRawMode() {
  tcgetattr(STDIN_FILENO, &orig_termios);
  atexit(disableRawMode);
  struct termios raw = orig_termios;
  raw.c_lflag &= ~(ECHO | ICANON);
  tcsetattr(STDIN_FILENO, TCSAFLUSH, &raw);
}

int main() { ... }
\end{mycode}

ICANON 来自 <termios.h>. 输入标志位(c\_iflag 中的标志位)通常以 I 开始, 例如 ICANON. 但是, ICANON 不是输入标志位, 它是 c\_lflag 字段中的``本地''标志位. 所以这很令人困惑.

现在程序将在你按下 q 时立即退出.

\section{显示按键}

为了更好地了解原始模式下的输入是如何工作的, 让我们在 read() 时, 打印出输入的每个字节. 我们将打印每个字符的 ASCII 值, 以及它所代表的字符(如果它是可打印字符).

\begin{mycode}{第8步: kilo.c}
#include <ctype.h>
#include <stdio.h>
#include <stdlib.h>
#include <termios.h>
#include <unistd.h>

struct termios orig_termios;

void disableRawMode() { ... }

void enableRawMode() { ... }

int main() {
  enableRawMode();

  char c;
  while (read(STDIN_FILENO, &c, 1) == 1 && c != 'q') {
    if (iscntrl(c)) {
      printf("%d\n", c);
    } else {
      printf("%d ('%c')\n", c, c);
    }
  }

  return 0;
}
\end{mycode}

iscntrl() 来自 <ctype.h>, printf() 来自 <stdio.h>.

iscntrl() 测试一个字符是否是一个控制字符. 控制字符是我们不想打印到屏幕上的不可打印字符. ASCII 码 0-31 都是控制字符, 127 也是控制字符. ASCII 码 32-126 都是可打印的. (查看 ASCII 表以查看所有字符.)

printf() 可以打印一个字节的多种表示形式. \%d 告诉它将字节格式化为十进制数(它的 ASCII 码), \%c 告诉它直接将字节作为一个字符输出.

这是一个非常有用的程序. 它向我们展示了各种按键如何转换为我们读取的字节. 大多数普通按键直接转换为它们所代表的字符. 但是试着看看当你按下 \textbf{方向键} 或者 \textbf{退出键}, 或者 \textbf{Page Up}, 或者 \textbf{Page Down}, 或者 \textbf{Home}, 或者 \textbf{End} , 或者 \textbf{退格键} , \textbf{删除键}, 或者 \textbf{回车键} 时会发生什么. 尝试使用带有 Ctrl 的组合键, 例如 Ctrl-A, Ctrl-B 等组合键.

你会注意到一些有趣的事情.

\begin{itemize}
\item 方向键, Page Up, Page Down, Home 和 End 会向终端输入 3 或 4 个字节: 27, '[', 然后是其他一两个字符. 这称为转义序列. 所有转义序列都以一个字节 27 开头. 按下 \textbf{退出键} 发送一个 27 字节作为输入.
\item \textbf{退格键} 是字节 127. Delete 键是一个 4 字节的转义序列.
\item \textbf{回车键} 是字节 10, 这是一个换行符, 也称为 '\\n'.
\item Ctrl-A 是 1, Ctrl-B 是 2, Ctrl-C 是...哦, 那程序就终止了, 对吧. 但 Ctrl 组合键似乎确实将字母 A-Z 映射到了代码 1-26 .
\end{itemize}

顺便说一句, 如果你碰巧按下 Ctrl-S, 你可能会发现你的程序似乎被冻结了. 你所做的是你已经要求你的程序停止向你发送输出. 按 Ctrl-Q 告诉它继续向你发送输出.

此外, 如果你按下 Ctrl-Z , 你的程序将暂停到后台. 运行 fg 命令将其带回前台.



\end{document}
